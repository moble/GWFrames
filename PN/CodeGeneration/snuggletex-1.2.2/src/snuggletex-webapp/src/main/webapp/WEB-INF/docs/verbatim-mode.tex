\pageId{verbatimMode}

\newenvironment{ndemotable}
{\begin{center}
 \begin{tabular}{|l|l|l|}
 \hline \\
 Input & Result & Notes \\
 \hline \\
}{\hline
 \end{tabular}
 \end{center}
}

\subsection*{Entering verbatim Mode}

SnuggleTeX supports the \verb|\verb| and \verb|\verb*| commands and the
\texttt{verbatim} environments in the normal way. These allow you to
enter ``verbatim'' text without having to escape it (see above), normally
displaying the results in a teletype font (though this is done via CSS
so you can override this if you want).

\begin{ndemotable}
\verb|\begin{verbatim}Hello There\end{verbatim}| &
\begin{verbatim}Hello There\end{verbatim} &
Content can span multiple lines \\
\hline
\verb!\verb|Hello There|! &
\verb|Hello There| &
You use a non-whitespace character to delimit the content,
which must end on the same line; whitespace within
the content is converted to a non-breaking space. \\
\hline
\verb!\verb*|Hello There|! &
\verb*|Hello There| &
Same as \verb|\verb| except that spaces are converted
to open boxes. \\
\hline
\verb|\literal{He said ?!@%#}| &
\literal{He said ?!@%#} &
This is specific to SnuggleTeX; it lets you enter
text without escaping but displays it as normal. \\
\end{ndemotable}

\subsection*{Creating Verbatim Mode Commands and Environments}

SnuggleTeX deviates from LaTeX in the way it allows Commands
and Environments to be defined that support verbatim mode.

Commands and Environments created using \verb|\newcommand|,
\verb|\newenvironment| and friends support verbatim mode insofar
as you can do things like:

\begin{verbatim}
\newcommand{\test}[1]{\verb|#1| is output as #1}
\test{$x$}
\end{verbatim}

This generates the output:

\begin{verbatim}
$x$ is output as x
\end{verbatim}

This is quite convenient --- especially when generating pages like these ---
but doesn't work the same way in LaTeX.

On other other hand, there is no way to indicate that certain arguments
of user-defined commands and environments should be themselves parsed in
verbatim mode, which makes it difficult to define macros out of existing
ones that do, such as the \verb|\href| command.

If you need this type of functionality, you can use the SnuggleTeX API to
define your own built-in commands or environment as these allow you to specify
which mode each argument (and environment content, if appropriate) should be
parsed in. The \verb|\literal| command is one very simple example of this
type of idea.
